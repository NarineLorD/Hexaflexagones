\documentclass[11pt,a4paper]{report}
\usepackage[utf8]{inputenc}
\usepackage{amsmath}
\usepackage{amsfonts}
\usepackage{amssymb}
\usepackage{graphicx}
\usepackage[left=2cm,right=2cm,top=2cm,bottom=2cm]{geometry}
\author{Raphaël Wormser}
\title{Construction d'hexaflexagones}
\begin{document}
	\maketitle
	\titlepage
	
	\tableofcontents
	
	\chapter{Introduction}
		\paragraph{Avertissement}
		Le nombre d'article dédié au sujet est relativement limité, les articles techniques francophones sont quasiment introuvable, il n'existe donc pas de terminologie arrêtée en français, ainsi certaines dénominations que j'utilise seront directement tirées des expression anglaises.
				
	\section{Présentation}
		\paragraph{}
		En 1939, le britannique Arthur Stone entra à l'université de Princeton, mais le papier américain était alors plus large que ses classeurs britanniques. Après avoir formaté ses pages, il se retrouva en possession de nombreuses bandes avec lesquelles il pu expérimenter divers pliage dont, entre autres, le tri-hexaflexagone.\\
		Même s'il ne s'agissait sans doute pas de la première découverte des flexagones, c'est Stone et ses camarades Bryant Tuckerman, John Tuckey, et Richard Feynman, qui pris d'engouement pour cette découverte ont commencé à les populariser.\\
		C'est en 1956, avec l'article de Martin Gardner dans {\it Scientific American}, que les hexaflexagones furent largement popularisés aux États-Unis
			
	
	
	\chapter{Représentations}
	\section{Lesquelles?}
	\paragraph{}
	L%'autre graphe associé au flexagone 



	\chapter{Résultat préliminaire}
	\section{Dénombrement des flexagones} 
	\paragraph*{}
		On doit compter le nombre de façon de trianguler un n-gone régulier.
		Soit la fonction suivante
		\[ 
			C:\ n \mapsto \left\{ \begin{array}{ll}
			         \frac{1}{n+1}\left( \begin{array}{ll}
			2n \\
			 n \end{array} \right) & \mbox{si $ n \in \mathbb{N}$}\\
			         0 & \mbox{sinon}.\end{array} \right.
		\]
		qui vaut le $n^{e}$ nombre de Catalan si n est entier, 0 partout ailleurs.

	\subsection{Identification aux nombres de catalan}
	Il est connu que le nombre de triangulations du n-gone régulier est exactement $C(n)$. Voici une explication succinte.\\	
	
	\section{Action du groupe diédral}
		\paragraph{}
			Il est clair que si deux triangulations se déduisent l'une de l'autre par une rotation ou une symétrie, elles définissent le même flexagone. Il est donc nécessaire de calculer $|F(n)/D_{n}|$ où $D_{n}$ est le groupe diédral d'ordre $2n$.
\end{document}