\documentclass[10pt,a4paper]{article}
\usepackage[utf8]{inputenc}
\usepackage[french]{babel}
\usepackage[T1]{fontenc}
\usepackage{graphicx}
\usepackage[left=2cm,right=2cm,top=2cm,bottom=2cm]{geometry}
\author{Raphaël Wormer}
\title{Hexaflexagones}


\begin{document}





\maketitle
\tableofcontents

\paragraph{Début}
C'est le début de mon document sur les hexaflexagones, j'espère (peut-être) que tout se passera bien.

\section{Introduction}
	\subsection{Motivations}
	\subsection{Objectifs}

\section{Hexaflexagones}
	\subsection{Présentation, historique}
	\subsection{Représentations}
		\subsubsection{Patrons}
		\subsubsection{Traversée de Tuckerman}
		\subsubsection{Diagramme de Tuckey}
		\subsubsection{Listes d'Oakley-Wisner}
	
\section{Triangulations de polygones}
	\subsection{Généralités}
	\subsection{Dénombrement}
		\subsubsection{Comptage totale}
		\subsubsection{Classes de symétrie}

\section{Programmation}
	\subsection{Structures de données}
	\subsection{•}





\end{document}