\documentclass[french,xcolor=dvipsnames]{beamer}
\usepackage[utf8]{inputenc}
\usepackage[T1]{fontenc}
\usepackage{amsmath}
\usepackage{amsfonts}
\usepackage{amssymb}
\usepackage{graphicx}
\usetheme{Antibes}
\usepackage{wrapfig}
\usecolortheme[named=Maroon]{structure}
\graphicspath{{Images/}}

\begin{document}
	\author{Raphaël Wormser}
	\title{Hexaflexagones}
	\subtitle{Construction et propriétés}
	%\subtitle{Constructions et propriétés}
	%\logo{}
	%\institute{}
	%\date{}
	%\subject{Courbes de largeur constantes}
	%\setbeamercovered{transparent}
	%\setbeamertemplate{navigation symbols}{}
	\frame[plain]{\maketitle}
	

\AtBeginSection
{
\frame{{Plan}\tableofcontents[currentsection]}
}

	\frame{{Plan}\tableofcontents}
	
	\section{I.Introduction}
		\subsection{I.1/Historique}
		\begin{frame}{Un bref historique}

			\begin{list}{Intro}{}
				\item[$\textit{1939}$.] Découverte  par Arthur Stone
				\item[$\textit{1956}$.]Colonne de Martin Gardner dans \textit{Scientific American}
				\item[Polulaire mais pas de succèes commercial]
			\end{list}
		\end{frame}
		
		
\end{document}