\documentclass[12pt,a4paper]{article}
\usepackage[latin1]{inputenc}
\usepackage{amsmath}
\usepackage{amsfonts}
\usepackage{amssymb}
\usepackage{graphicx}
\usepackage[left=1.00cm, right=1.00cm]{geometry}
\title{Hexaflexagone: Programme d'assistance � la construction}

\begin{document}
	\begin{flushleft}
		\textbf{\underline{\Huge{Hexaflexagones:}}}\newline\newline
		\textbf{\underline{\huge{Programme d'assistance � la construction }}}
	\end{flushleft}

	\begin{center}
		Rapha�l Wormser\\
	\end{center}
	
	\paragraph{Remarques}
	Il �tait une fois dans l'ouest\\
	
	\paragraph{\textbf{\underline{\large{Th�matiques}}}}
	Combinatoire, programmation dynamique, algorithmique
	
		
	
	\bibliography{*}
	
	
	
	\begin{thebibliography}{}	
		\bibitem{}\textit{Counting Symetry classes of a convex regular polygon}, \textsc{Douglas Bownman \& Alon Regev} (2012)
		
		\bibitem{DKing} \textit{A bit of flexagon theory}, \textsc{David King} (1999)
		\bibitem{}\textit{}, \textsc{} ()
	\end{thebibliography}
\end{document}